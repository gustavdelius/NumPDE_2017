\documentclass{beamer}

\mode<presentation>


\usepackage{amsmath,amssymb,amsfonts,amsbsy,amsthm}
\usepackage{epsfig,array}
\usepackage[hidelinks]{hyperref}

\theoremstyle{plain}
\newtheorem{theorem}{Theorem}[section]
\newtheorem{algorithm}[theorem]{Algorithm}
\newtheorem{claim}[theorem]{Claim}
\newtheorem{conclusion}{Conclusion}
\newtheorem{condition}{Condition}
\newtheorem{conjecture}{Conjecture}
\newtheorem{corollary}[theorem]{Corollary}
\newtheorem{criterion}{Criterion}
\newtheorem{lemma}[theorem]{Lemma}
\newtheorem*{lemma*}{Lemma}
\newtheorem{prop}[theorem]{Proposition}

\theoremstyle{definition}
\newtheorem{definition}[theorem]{Definition}
\newtheorem{example}{Example}[section]
\newtheorem*{example*}{Example}
\newtheorem{exercise}[example]{Problem}
\newtheorem{problem}[example]{Problem}
\newtheorem{remark}[theorem]{Remark}
\newtheorem{ruleofthumb}[theorem]{Rule of thumb}
\newtheorem*{remark*}{Remark}
\newtheorem{reminder}[theorem]{Reminder}
%\newtheorem*{solution}{Solution}
\newtheorem{summary}{Summary}
%\newenvironment{proof}[1][Proof]{\textbf{#1.} }{\ \rule{0.5em}{0.5em}}
\newenvironment{solution}[1][{\bf Solution}]{ \emph{#1.} }{\medskip}
\newenvironment{notation}[1][\textbf{Notation:} ]{  #1}{\medskip}

\numberwithin{figure}{section}
\numberwithin{equation}{section}

% Definitions
%================================================================
\def\curl{{\rm curl}\,}
\def\div{{\rm div}\,}
\def\la{\lambda}
\def\ba{{\bf a}}
\def\bb{{\bf b}}
\def\be{{\bf e}}
\def\bj{{\bf j}}
\def\bn{{\bf n}}
\def\bV{{\bf V}}
\def\bJ{{\bf J}}
\def\bv{{\bf v}}
\def\bu{{\bf u}}
\def\bH{{\bf H}}
\def\vf{{\bf f}}
\def\bh{{\bf h}}
\def\bU{{\bf U}}
\def\bV{{\bf V}}
\def\bx{{\bf x}}
\def\by{{\bf y}}
\def\bz{{\bf z}}
\def\bX{{\bf X}}
\def\bC{{\bf C}}

\def\vx {{\bf x}}
\def\vy {{\bf y}}
\def\vz {{\bf z}}
\def\vv {{\bf v}}
\def\vV {{\bf V}}
\def\vu {{\bf u}}
\def\vb {{\bf b}}
\def\vc {{\bf c}}
\def\vr {{\bf r}}

\def\pr{{\partial}}
\def\eps{{\epsilon}}
\def\veta{\boldsymbol{\eta}}
\def\bxi{\boldsymbol{\xi}}
\def\bnu{\boldsymbol{\nu}}
\def\Bom{\boldsymbol{\Omega}}
\def\pd#1#2{\frac{\displaystyle\partial#1}{\displaystyle\partial#2}}
\def\bfr#1#2{\frac{\displaystyle #1}{\displaystyle #2}}
\def\vec#1{\boldsymbol{#1}}
\def\Bbb{\mathbb}
\def \shalf{{\textstyle \frac{1}{2}}}
\def \half{\frac{1}{2}}
\def \ssum{{\textstyle \sum}}



\usepackage[normalem]{ulem}

\begin{document}

\setbeamercovered{transparent}

%%%%%%%%%%%%%%%%%%%%%%%%%%%%%%%%%%%%%%%%%%%%%%%%%%%%%%%%%%%%%%%%%%%%%%%%%%%%%%%%%%%%%%%%%%%%%%%%%%%%%%%%%%%%%%%%%%%

\begin{frame}{Numerical Methods for PDEs (Spring 2017)}

\begin{itemize}

\item{} \textcolor{blue}{\textbf{Module coordinators}}:

Gustav Delius and Richard Southwell

\item{} \textcolor{blue}{\textbf{Module aims}}:

\begin{itemize}

\item{} To gain an understanding of basic numerical methods for solving PDEs

\item{} To be aware of the potential pitfalls and develop experience in 
avoiding them

\item{} To implement numerical algorithms in practice

\item{} To gain competency in using computing tools like R, C++, Git, ...

\end{itemize}

\end{itemize}

\end{frame}

%%%%%%%%%%%%%%%%%%%%%%%%%%%%%%%%%%%%%%%%%%%%%%%%%%%%%%%%%%%%%%%%%%%%%%%%%%%%%%%%%%%%%%%%%%%%%%%%%%%%%%%%%%%%%%%%%%%

\begin{frame}{Teaching and assessment}

\begin{itemize}

\item{} \textcolor{blue}{\textbf{Teaching}}:

\begin{itemize}

\item<1|only@1>{} 18 hours of lectures

\item<1|only@1> 9 hours of practicals (computer classes)

\item<2|only@2->{\sout{18 hours of lectures}}

\item<2|only@2->{\sout{9 hours of practicals (computer classes)}}

\item<2-|only@2-> 27 hours of discussion of theory and computer work
\begin{itemize}
\item<3-> Lecture notes
\item<3-> Lecture slides and summary sheets
\item<3-> Problem and solution sheets
\item<3-> Computer worksheets with example code and exercises
\end{itemize}

\end{itemize}
\pause
\item{} \textcolor{blue}{\textbf{Assessment}}: 

\begin{itemize}

\item<4-> 2 hour exam in week 1 of the
Summer term -- 75\% 

\item<5-> Coursework -- 25\%

\begin{itemize}

\item<6-> mini-project 1 (10\%) Implementing a numerical scheme for solving a given parabolic equation

\item<7-> mini-project 2 (15\%) Implementing a numerical scheme for a PDE from a published research paper of your choice

\end{itemize}

\end{itemize}

\end{itemize}



\end{frame}

%%%%%%%%%%%%%%%%%%%%%%%%%%%%%%%%%%%%%%%%%%%%%%%%%%%%%%%%%%%%%%%%%%%%%%%%%%%%%%%%%%%%%%%%%%%%%%%%%%%%%%%%%%%%%%%%%%%

\begin{frame}{Course contents}

{\footnotesize

\begin{enumerate}

\item{} Introduction (1 week).

\item{} Parabolic differential equations. Finite-difference methods. Consistency, stability and convergence of
finite-difference schemes. (5 weeks).

\item{} Finite-difference methods for elliptic differential equations (2 weeks).

\item{} Finite-difference methods for hyperbolic differential equations (1 week).

\end{enumerate}

\vskip 5mm
\pause

\textcolor{blue}{\textbf{Reading list}}:

\begin{enumerate}

\item{} RL Burden \& JD Faires, {\it Numerical Analysis}
(6th ed.), Brooks/Cole Publishing Company, 1997;

\item{} WF Ames,
{\it Numerical Methods for Partial Differential Equations}, Academic
Press, 1977;

\item{} WH Press,
{\it Numerical Recipes: the Art of Scientific Computing}, CUP, 2007.

\end{enumerate}
}

\end{frame}



%%%%%%%%%%%%%%%%%%%%%%%%%%%%%%%%%%%%%%%%%%%%%%%%%%%%%%%%%%%%%%%%%%%%%%%%%%%%%%%%%%%%%%%%%%%%%%%%%%%%%%%%%%%%%%%%%%%



\begin{frame}{\small Some facts from Calculus}

{\footnotesize

\textbf{Taylor's theorem for functions of one variable}

\vskip 3mm
If $f\in C^{n+1}$ in $(x_{0}-\epsilon,x_{0}+\epsilon)$, then
\[
f(x)=T_{n}+R_{n}
\]
where $T_{n}$ in the $n$th Taylor polynomial
\[
T_{n}=f(x_{0})+(x-x_{0})f^{\prime}(x_{0})+
\frac{(x-x_{0})^{2}}{2!}f^{\prime\prime}(x_{0})+ \dots+
\frac{(x-x_{0})^{n}}{n!}f^{(n)}(x_{0})
\]
and $R_{n}$ is the remainder term
\[
R_{n}=\frac{(x-x_{0})^{n+1}}{(n+1)!}f^{(n+1)}(\xi)
\]
for some point $\xi$ between $x_{0}$ and $x$ ($\xi$ can be written
as $\xi=x_{0}+\theta (x-x_{0})$ where $0< \theta<1$).

}

\end{frame}

%%%%%%%%%%%%%%%%%%%%%%%%%%%%%%%%%%%%%%%%%%%%%%%%%%%%%%%%%%%%%%%%%%%%%%%%%%%%%%%%%%%%%%%%%%%%%%%%%%%%%%%%%%%%%%%%%%%


\begin{frame}{\small Some facts from Calculus}

{\footnotesize

\textbf{Taylor's theorem for functions of two variables}

\vskip 3mm
If $F\in C^{n+1}(D)$ where $D\subset\mathbb{R}^2$,
then
\begin{eqnarray}
F(x, y) &=& T_{n}(x, y)+R_{n}(x, y), \nonumber \\
T_{n}(x, y) &=& F(x_{0}, y_{0})+
\left[\Delta x\left.\frac{\partial F}{\partial x}\right\vert_{(x_{0}, y_{0})}+
\Delta y\left.\frac{\partial F}{\partial y}\right\vert_{(x_{0}, y_{0})}\right] \nonumber \\
&+& \left[\frac{(\Delta x)^{2}}{2}\left.\frac{\partial^2 F}{\partial x^2}\right\vert_{(x_{0}, y_{0})}+
\Delta x \Delta y\left.\frac{\partial^2 F}{\partial x\partial y}\right\vert_{(x_{0}, y_{0})}+
\frac{(\Delta y)^{2}}{2}\left.\frac{\partial^2 F}{\partial y^2}\right\vert_{(x_{0}, y_{0})}
\right] \qquad \nonumber \\
&+& \dots\dots \nonumber \\
&+& \left[\frac{1}{n!}\sum_{j=0}^{n}
\left(
\begin{array}{c}
n \\
 j
\end{array}
\right)
(\Delta x)^{n-j}(\Delta y)^{j}
\left.\frac{\partial^{n} F}{\partial x^{n-j}\partial y^{j}}\right\vert_{(x_{0}, y_{0})}\right] \nonumber
\end{eqnarray}
where $\Delta x=x-x_{0}$, $\Delta y=y-y_{0}$ and
\[
R_{n}(x, y)=
\frac{1}{(n+1)!}\sum_{j=0}^{n+1}
\left(
\begin{array}{c}
n+1 \\
 j
\end{array}
\right)
(\Delta x)^{n+1-j}(\Delta y)^{j}
\frac{\partial^{n+1} f}{\partial x^{n+1-j}\partial y^{j}}(\xi, \mu) .
\]
}



\end{frame}

%%%%%%%%%%%%%%%%%%%%%%%%%%%%%%%%%%%%%%%%%%%%%%%%%%%%%%%%%%%%%%%%%%%%%%%%%%%%%%%%%%%%%%%%%%%%%%%%%%%%%%%%%%%%%%%%%%%


\begin{frame}{\small Some facts from Calculus}



\textbf{Big $O$ notation}

\vskip 5mm
\textcolor{blue}{\textbf{Definition.}} Let $\lim\limits_{x\to 0}g(x)=0$ and
$\lim\limits_{x\to 0}f(x)=f_{0}$. If there exists
a constant $K>0$ such that
\[
\vert f(x)-f_{0}\vert   \leq K \vert g(x)\vert,
\]
at least for $x$ sufficiently close to zero, we write
\[
f(x)=f_{0}+O(g(x)) \quad {\rm as} \quad x\to 0.
\]


\end{frame}

%%%%%%%%%%%%%%%%%%%%%%%%%%%%%%%%%%%%%%%%%%%%%%%%%%%%%%%%%%%%%%%%%%%%%%%%%%%%%%%%%%%%%%%%%%%%%%%%%%%%%%%%%%%%%%%%%%%


\begin{frame}{\small Some facts from Calculus}


\begin{itemize}[<+->]

\item
\textcolor{blue}{\textbf{Example}}. Let us show that $\sin(x)/x=1+O(x^2)$ as $x\to 0$.


\vskip 3mm

\item \textbf{Solution.}
The 2nd Taylor polynomial for $sin(x)$:
\[
\sin(x)=x-\frac{x^{3}}{3!}\cos(\xi)
\]
where $\xi$ is some number between 0 and $x$.

\vskip 3mm
We have
\[
\left\vert \frac{\sin x}{x} -1 \right\vert = \frac{\vert x
\vert^{2}}{3!}\vert \cos(\xi)\vert \leq \frac{x^{2}}{3!}=\frac{x^{2}}{6} \ \ \
\Rightarrow \ \ \ \ \frac{\sin x}{x}=1+O(x^{2}).
\]

\end{itemize}


\end{frame}

%%%%%%%%%%%%%%%%%%%%%%%%%%%%%%%%%%%%%%%%%%%%%%%%%%%%%%%%%%%%%%%%%%%%%%%%%%%%%%%%%%%%%%%%%%%%%%%%%%%%%%%%%%%%%%%%%%%


\begin{frame}{\small Some facts from Calculus}

\textcolor{blue}{\textbf{Properties of $O(x^n)$}}:

\begin{enumerate}[<+->]
\item{} $O(x^n)+O(x^m)=O(x^k)$ \ \ for $n,m\geq 0$ \ \ and \ \ $k=\min\{n,m\}$.

\vskip 3mm
\item{} $O(x^n)O(x^m)=O(x^{n+m})$ \ \ for \ \ $n,m\geq 0$.

\vskip 3mm
\item{} $x^m O(x^n)=O(x^{n+m})$ \ \ for $n\geq 0$ \ \ and \ \ $n+m\geq 0$.
\end{enumerate}



\end{frame}

\end{document}