\documentclass[10pt]{article}

\usepackage{amsmath,amssymb,amsfonts,amsbsy}
\usepackage{epsfig,array}


\textwidth 17cm \textheight 24cm \setlength{\oddsidemargin}{-3mm}
\setlength{\evensidemargin}{-3mm}
\setlength{\headheight}{-1\baselineskip}
\setlength{\headsep}{-1\baselineskip}

\renewcommand{\section}{\subsection}

\pagestyle{empty}



% Definitions
%================================================================
\def\curl{{\rm curl}\,}
\def\div{{\rm div}\,}
\def\la{\lambda}
\def\ba{{\bf a}}
\def\bb{{\bf b}}
\def\be{{\bf e}}
\def\bj{{\bf j}}
\def\bn{{\bf n}}
\def\bV{{\bf V}}
\def\bJ{{\bf J}}
\def\bv{{\bf v}}
\def\bu{{\bf u}}
\def\bH{{\bf H}}
\def\vf{{\bf f}}
\def\bh{{\bf h}}
\def\bU{{\bf U}}
\def\bV{{\bf V}}
\def\bx{{\bf x}}
\def\by{{\bf y}}
\def\bz{{\bf z}}
\def\bX{{\bf X}}
\def\bC{{\bf C}}

\def\vx {{\bf x}}
\def\vy {{\bf y}}
\def\vz {{\bf z}}
\def\vv {{\bf v}}
\def\vu {{\bf u}}
\def\vb {{\bf b}}
\def\vc {{\bf c}}
\def\vr {{\bf r}}

\def\pr{{\partial}}
\def\veta{\boldsymbol{\eta}}
\def\bxi{\boldsymbol{\xi}}
\def\bnu{\boldsymbol{\nu}}
\def\Bom{\boldsymbol{\Omega}}
\def\pd#1#2{\frac{\displaystyle\partial#1}{\displaystyle\partial#2}}
\def\bfr#1#2{\frac{\displaystyle #1}{\displaystyle #2}}
\def\vec#1{\boldsymbol{#1}}
\def\Bbb{\mathbb}
\def \shalf{{\textstyle \frac{1}{2}}}
\def \half{\frac{1}{2}}
\def \ssum{{\textstyle \sum}}

%=======================================================================

\begin{document}

\begin{center}
{\large{\bf Numerical Methods for PDEs (Spring 2017)}}
\end{center}

\begin{center}
{\large{\bf Problems 2}}
\end{center}

\noindent
{\bf Hand in written solutions to problems 6 and 7 at the start of the lecture on 21 February.}

\noindent
Consider the heat equation
\begin{equation}
u_{t}-K u_{xx}=0 \quad \hbox{for} \quad 0<x<1, \ \ t>0, \label{1}
\end{equation}
subject to the boundary conditions
\begin{equation}
u(0,t)=0, \quad u(1, t)=0,  \label{2}
\end{equation}
and the initial condition
\begin{equation}
u(x,0)=u_{0}(x).  \label{3}
\end{equation}

\vskip 0.5cm \noindent
{\bf Problem 4.} Show that the Du Fort - Frankel method for Eq. (\ref{1}),
given by
\[
\frac{w_{k,j+1}-w_{k,j-1}}{2\tau}-K \frac{w_{k+1,
j}-w_{k,j-1}-w_{k,j+1}+w_{k-1,j}}{h^{2}}=0,
\]
has the local truncation error $O\left(\tau^2+h^2+\tau^2/h^2\right)$.

\vskip 0.5cm \noindent
{\bf Problem 5.} The initial boundary value
problem (\ref{1})--(\ref{3}) is solved numerically using the
finite-difference method:
\begin{eqnarray}
&&w_{k0}=u_{0}(x_{k}), \quad w_{0j}=0, \quad w_{Nj}=0,   \nonumber \\
&&\frac{w_{k,j+1}-w_{k,j}}{\tau}-K(1-\sigma) \frac{w_{k+1,
j}-2w_{kj}+w_{k-1,j}}{h^{2}}-K\sigma \frac{w_{k+1,
j+1}-2w_{k,j+1}+w_{k-1,j+1}}{h^{2}}=0,   \label{4}
\end{eqnarray}
for $k=1, 2, \dots , N-1$ and $j=0, 1, \dots$. Here $w_{kj}$ is an
approximation to $u(x_{k}, y_{j})$ and $x_{k}=k h$
$(k=0,1,\dots,N)$, $t_{j}=j \tau$ $(j=0,1,\dots)$, $h=\frac{1}{N}$.
In Eqs. (\ref{4}), $\sigma$ is a real parameter such that
$0\leq \sigma\leq 1$. Show that the method is stable if
\[
\sigma\geq\frac{1}{2}\left(1-\frac{1}{2\gamma}\right)
\]
where $\gamma=K\tau/h^2$.


\vskip 0.5cm
\noindent
{\bf Problem 6.} Show that if
\[
\gamma\equiv\frac{K\tau}{h^2}=\frac{1}{6}
\]
in the explicit forward-difference method for Eq. (\ref{1}):
\[
\frac{w_{k,j+1}-w_{kj}}{\tau}-K
\frac{w_{k+1, j}-2w_{kj}+w_{k-1,j}}{h^{2}}=0,
\]
then the local truncation error is $O(\tau^2)$ or, equivalently $O(h^4)$.


\vskip 0.5cm
\noindent
{\bf Problem 7.}
Devise a backward difference scheme of $O(\tau+h^2)$ for the non-homogeneous heat
equation eq.(2.73) in the notes with boundary conditions as given in eq.(2.80). This means
you need to derive an expression for $w_{0,j+1}$ similar to eq.(2.78) and also a similar
expression for $w_{N,j+1}$.

\vskip 0.5cm
\noindent
{\bf Problem 8.}
Consider the equation
\[
\frac{\partial u}{\partial t}=3\frac{\partial^{2}u}{\partial x^{2}}-
2a(x,t)\frac{\partial u}{\partial x}+b(x,t) \quad \hbox{for} \quad
0<x<1, \ \ t > 0,
\]
subject to the initial and boundary conditions
\[
u(0,t)=\mu_{1}(t), \quad u(1, t)=\mu_{2}(t),
\quad u(x, 0)=u_{0}(x) .
\]
Obtain a finite-difference approximation to this boundary-value
problem and show that your finite-difference method is consistent with the equation,
i.e. that the local truncation errors tend to zero as step sizes in $x$ and in $t$
go to zero.


\end{document}
